\documentclass{article}

\usepackage{pvs} % problems in pvs: uses amsopn,
                 % which defines conflicting theorem/lemma etc
\usepackage{latexsym}
\usepackage{amsopn}
\usepackage{amsfonts}
\usepackage{alltt}

\begin{document}

\section{Introduction}

This set of libraries defines the extended nonnegative reals ({\it
  i.e} the nonnegative reals with $\infty$) as a boolean/nonnegative
real tuple. If one can get past the cluttered notation ({\it e.g.}
\pvsid{x\_eq} for $=$), it actually works quite well in practice.

The primary purpose of this library is to support measure theory, so
the operations supported are quite restricted. Notice that there is no
\pvsid{x\_ge}, \pvsid{x\_gt} defined. However, we {\em are} interested
in sums and limits, including least upper bounds and greatest lower
bounds.

There are three supporting libraries, which may be of more general
interest. The first is \pvsid{code\_product}, which performs a
diagonal traversal of a doubly indexed sequence. A nice way to use
this feature is shown in \pvsid{double\_index}, where the deft use of
\pvsid{double\_index} turns a sequence into a doubly-indexed function,
and \pvsid{single\_index} turns a doubly-indexed function back into a
sequence.

The final supporting library -- \pvsid{double\_nn\_sequence} allows us
to manipulate double sums. It relies (obviously) on the items being
manipulated being non-negative. Put simply, one needs to pay a great
deal of attention to re-ordering infinite sums if the series are not
absolutely convergent. If someone wants to put in the work to
generalize this result to more general series, feel free. I'd suggest
using the result for non-negative sums as a starting point.

\appendix
\newpage
\section{Library extended\_nnreal}
% The following substitutions are from the file:
%   /home/david/PVS42/pvs-tex.sub
\def\xunderscorelttwofn#1#2{{#1 < #2}}% How to print function x_lt with arity (2)
\def\xunderscoreletwofn#1#2{{#1 \leq #2}}% How to print function x_le with arity (2)
\def\xunderscoreeqtwofn#1#2{{#1 = #2}}% How to print function x_eq with arity (2)
\def\xunderscoreaddtwofn#1#2{{#1 + #2}}% How to print function x_add with arity (2)
\def\xunderscorelimitonefn#1{{\pvsid{limit}(#1)}}% How to print function x_limit with arity (1)
\def\xunderscoresumonefn#1{{\sum #1}}% How to print function x_sum with arity (1)
\def\xunderscoresigmathreefn#1#2#3{{\sum_{#1}^{#2} #3}}% How to print function x_sigma with arity (3)
\def\xunderscoresuponefn#1{{\pvsid{sup}(#1)}}% How to print function x_sup with arity (1)
\def\xunderscoreinfonefn#1{{\pvsid{inf}(#1)}}% How to print function x_inf with arity (1)
\def\Etwofn#1#2{{\mathbb{E}(#1~|~#2)}}% How to print function E with arity (2)
\def\Eonefn#1{{\mathbb{E}(#1)}}% How to print function E with arity (1)
\def\Ptwofn#1#2{{\mathbb{P}(#1~|~#2)}}% How to print function P with arity (2)
\def\Ponefn#1{{\mathbb{P}(#1)}}% How to print function P with arity (1)
\def\xtwofn#1#2{{#1\times#2}}% How to print function x with arity (2)
\def\asttwofn#1#2{{#1\ast#2}}% How to print function ast with arity (2)
\def\astonefn#1{{{#1}^{\ast}}}% How to print function ast with arity (1)
\def\dottwofn#1#2{{#1\bullet#2}}% How to print function dot with arity (2)
\def\integralthreefn#1#2#3{{\int_{#1}^{#2} #3}}% How to print function integral with arity (3)
\def\integraltwofn#1#2{{\int_{#1} #2}}% How to print function integral with arity (2)
\def\integralonefn#1{{\int#1}}% How to print function integral with arity (1)
\def\normonefn#1{{\left||{#1}\right||}}% How to print function norm with arity (1)
\def\phionefn#1{{\pvssubscript{\phi}{#1}}}% How to print function phi with arity (1)
\def\infunderscoreclosedonefn#1{{\left(-\infty,~#1\right]}}% How to print function inf_closed with arity (1)
\def\closedunderscoreinfonefn#1{{\left[#1,~\infty\right)}}% How to print function closed_inf with arity (1)
\def\infunderscoreopenonefn#1{(-\infty,~#1)}% How to print function inf_open with arity (1)
\def\openunderscoreinfonefn#1{(#1,~\infty)}% How to print function open_inf with arity (1)
\def\closedtwofn#1#2{{\left[#1,~#2\right]}}% How to print function closed with arity (2)
\def\opentwofn#1#2{(#1,~#2)}% How to print function open with arity (2)
\def\sigmathreefn#1#2#3{{\sum_{#1}^{#2} #3}}% How to print function sigma with arity (3)
\def\sigmatwofn#1#2{{\sum_{#1} {#2}}}% How to print function sigma with arity (2)
\def\ceilingonefn#1{{\lceil{#1}\rceil}}% How to print function ceiling with arity (1)
\def\flooronefn#1{{\lfloor{#1}\rfloor}}% How to print function floor with arity (1)
\def\absonefn#1{{\left|{#1}\right|}}% How to print function abs with arity (1)
\def\roottwofn#1#2{{\sqrt[#2]{#1}}}% How to print function root with arity (2)
\def\sqrtonefn#1{{\sqrt{#1}}}% How to print function sqrt with arity (1)
\def\sqonefn#1{{\pvssuperscript{#1}{2}}}% How to print function sq with arity (1)
\def\expttwofn#1#2{{\pvssuperscript{#1}{#2}}}% How to print function expt with arity (2)
\def\opcarettwofn#1#2{{\pvssuperscript{#1}{#2}}}% How to print function ^ with arity (2)
\def\indexedunderscoresetsotherIIntersectiononefn#1{{\bigcap #1}}% How to print function indexed_sets.IIntersection with arity (1)
\def\indexedunderscoresetsotherIUniononefn#1{{\bigcup #1}}% How to print function indexed_sets.IUnion with arity (1)
\def\setsotherIntersectiononefn#1{{\bigcap #1}}% How to print function sets.Intersection with arity (1)
\def\setsotherUniononefn#1{{\bigcup #1}}% How to print function sets.Union with arity (1)
\def\setsotherremovetwofn#1#2{{(#2 \setminus \{#1\})}}% How to print function sets.remove with arity (2)
\def\setsotheraddtwofn#1#2{{(#2 \cup \{#1\})}}% How to print function sets.add with arity (2)
\def\setsotherdifferencetwofn#1#2{{(#1 \setminus #2)}}% How to print function sets.difference with arity (2)
\def\setsothercomplementonefn#1{{\overline{#1}}}% How to print function sets.complement with arity (1)
\def\setsotherintersectiontwofn#1#2{{(#1 \cap #2)}}% How to print function sets.intersection with arity (2)
\def\setsotheruniontwofn#1#2{{(#1 \cup #2)}}% How to print function sets.union with arity (2)
\def\setsotherstrictunderscoresubsetothertwofn#1#2{{(#1 \subset #2)}}% How to print function sets.strict_subset? with arity (2)
\def\setsothersubsetothertwofn#1#2{{(#1 \subseteq #2)}}% How to print function sets.subset? with arity (2)
\def\setsothermembertwofn#1#2{{(#1 \in #2)}}% How to print function sets.member with arity (2)
\def\opohtwofn#1#2{{#1 \circ #2}}% How to print function O with arity (2)
\def\opdividetwofn#1#2{{\frac{#1}{#2}}}% How to print function / with arity (2)
\def\optimestwofn#1#2{{#1\times#2}}% How to print function * with arity (2)
\def\opdifferenceonefn#1{{-#1}}% How to print function - with arity (1)
\def\opdifferencetwofn#1#2{{#1-#2}}% How to print function - with arity (2)
\def\opplustwofn#1#2{{#1+#2}}% How to print function + with arity (2)
\begin{alltt}
\({\pvssubscript{\overline{\mathbb{R}}}{{\geq}0}}\): \pvskey{THEORY}
 \pvskey{BEGIN}

  \pvskey{IMPORTING} \pvsid{sigma\_set}@\pvsid{series\_aux}, \pvsid{sigma\_set}@\pvsid{absconv\_series}, \pvsid{reals}@\(\sigma\)\({\pvsbracketl}\)\(\mathbb{N}\)\({\pvsbracketr}\), \pvsid{lester}@\pvsid{varying\_epsilon}

  \pvsid{limit}: \pvskey{MACRO} \({\pvsbracketl}\)\pvsid{(}\pvsid{convergent}\pvsid{)} \(\rightarrow\) \(\mathbb{R}\)\({\pvsbracketr}\) \pvskey{=} \pvsid{convergence\_sequences}.\pvsid{limit};\vspace*{\pvsdeclspacing}

  \({\pvssubscript{\overline{\mathbb{R}}}{{\geq}0}}\): \pvskey{TYPE+} = \({\pvsbracketl}\)\pvsid{bool}, \({\pvssubscript{\mathbb{R}}{{\geq}0}}\)\({\pvsbracketr}\) \pvskey{CONTAINING} \pvsid{(}\({\pvskey{TRUE}}\), \(0\)\pvsid{)}\vspace*{\pvsdeclspacing}

  \(i\), \(j\), \pvsid{low}, \pvsid{high}: \pvskey{VAR} \(\mathbb{N}\)\vspace*{\pvsdeclspacing}

  \(x\), \(y\), \(x\sb{1}\), \(y\sb{1}\), \(x\sb{2}\), \(y\sb{2}\): \pvskey{VAR} \({\pvssubscript{\overline{\mathbb{R}}}{{\geq}0}}\)\vspace*{\pvsdeclspacing}

  \(z\): \pvskey{VAR} \(\mathbb{R}\)\vspace*{\pvsdeclspacing}

  \(S\), \(T\): \pvskey{VAR} \({\pvsbracketl}\)\(\mathbb{N}\) \(\rightarrow\) \({\pvssubscript{\overline{\mathbb{R}}}{{\geq}0}}\)\({\pvsbracketr}\)\vspace*{\pvsdeclspacing}

  \(X\): \pvskey{VAR} \pvsid{set}\({\pvsbracketl}\)\({\pvssubscript{\overline{\mathbb{R}}}{{\geq}0}}\)\({\pvsbracketr}\)\vspace*{\pvsdeclspacing}

  \(\varepsilon\): \pvskey{VAR} \({\pvssubscript{\mathbb{R}}{{>}0}}\)\vspace*{\pvsdeclspacing}

  \(c\): \pvskey{VAR} \({\pvssubscript{\mathbb{R}}{{\geq}0}}\)\vspace*{\pvsdeclspacing}

  \(\xunderscoreinfonefn{X}\): \({\pvssubscript{\overline{\mathbb{R}}}{{\geq}0}}\) \pvskey{=}
      \pvskey{IF} \pvsid{(}\(\forall\) \pvsid{(}\(x\): \pvsid{(}\(X\)\pvsid{)}\pvsid{)}: \(\neg\) \(x\)`\(1\)\pvsid{)}
        \pvskey{THEN} \pvsid{(}\({\pvskey{FALSE}}\), \(0\)\pvsid{)}
      \pvskey{ELSE} \pvsid{(}\({\pvskey{TRUE}}\), \pvsid{glb}\pvsid{(}\{\(z\) | \(\exists\) \(x\): \(X\)\pvsid{(}\(x\)\pvsid{)} \(\wedge\) \(x\)`\(1\) \(\wedge\) \(x\)`\(2\) \(=\) \(z\)\}\pvsid{)}\pvsid{)}
      \pvskey{ENDIF}\vspace*{\pvsdeclspacing}

  \(\xunderscoresuponefn{X}\): \({\pvssubscript{\overline{\mathbb{R}}}{{\geq}0}}\) \pvskey{=}
      \pvskey{IF} \pvsid{empty?}\pvsid{(}\(X\)\pvsid{)}
        \pvskey{THEN} \pvsid{(}\({\pvskey{TRUE}}\), \(0\)\pvsid{)}
      \pvskey{ELSIF} \pvsid{(}\(\exists\) \pvsid{(}\(x\): \pvsid{(}\(X\)\pvsid{)}\pvsid{)}: \(\neg\) \(x\)`\(1\)\pvsid{)} \(\vee\)
             \(\neg\) \pvsid{bounded\_above?}\pvsid{(}\{\(z\) | \(\exists\) \(x\): \(X\)\pvsid{(}\(x\)\pvsid{)} \(\wedge\) \(x\)`\(1\) \(\wedge\) \(x\)`\(2\) \(=\) \(z\)\}\pvsid{)}
        \pvskey{THEN} \pvsid{(}\({\pvskey{FALSE}}\), \(0\)\pvsid{)}
      \pvskey{ELSE} \pvsid{(}\({\pvskey{TRUE}}\), \pvsid{lub}\pvsid{(}\{\(z\) | \(\exists\) \(x\): \(X\)\pvsid{(}\(x\)\pvsid{)} \(\wedge\) \(x\)`\(1\) \(\wedge\) \(x\)`\(2\) \(=\) \(z\)\}\pvsid{)}\pvsid{)}
      \pvskey{ENDIF}\vspace*{\pvsdeclspacing}

  \(\xunderscoreinfonefn{S}\): \({\pvssubscript{\overline{\mathbb{R}}}{{\geq}0}}\) \pvskey{=} \(\xunderscoreinfonefn{\pvsid{image}\pvsid{(}S, \pvsid{fullset}{\pvsbracketl}\mathbb{N}{\pvsbracketr}\pvsid{)}}\)\vspace*{\pvsdeclspacing}

  \(\xunderscoresuponefn{S}\): \({\pvssubscript{\overline{\mathbb{R}}}{{\geq}0}}\) \pvskey{=} \(\xunderscoresuponefn{\pvsid{image}\pvsid{(}S, \pvsid{fullset}{\pvsbracketl}\mathbb{N}{\pvsbracketr}\pvsid{)}}\)\vspace*{\pvsdeclspacing}

  \(\xunderscoresigmathreefn{\pvsid{low}}{\pvsid{high}}{S}\): \({\pvssubscript{\overline{\mathbb{R}}}{{\geq}0}}\) \pvskey{=}
      \pvskey{IF} \pvsid{(}\(\forall\) \(i\): \pvsid{low} \(\leq\) \(i\) \(\wedge\) \(i\) \(\leq\) \pvsid{high} \(\Rightarrow\) \(S\)\pvsid{(}\(i\)\pvsid{)}`\(1\)\pvsid{)}
        \pvskey{THEN} \pvsid{(}\({\pvskey{TRUE}}\), \(\sigmathreefn{\pvsid{low}}{\pvsid{high}}{\lambda i: S\pvsid{(}i\pvsid{)}`2}\)\pvsid{)}
      \pvskey{ELSE} \pvsid{(}\({\pvskey{FALSE}}\), \(0\)\pvsid{)}
      \pvskey{ENDIF}\vspace*{\pvsdeclspacing}

  \(\xunderscoresumonefn{S}\): \({\pvssubscript{\overline{\mathbb{R}}}{{\geq}0}}\) \pvskey{=}
      \pvskey{IF} \pvsid{(}\(\forall\) \(i\): \(S\)\pvsid{(}\(i\)\pvsid{)}`\(1\)\pvsid{)} \(\wedge\) \pvsid{convergent}\pvsid{(}\pvsid{series}\pvsid{(}\(\lambda\) \(i\): \(S\)\pvsid{(}\(i\)\pvsid{)}`\(2\)\pvsid{)}\pvsid{)}
        \pvskey{THEN} \pvsid{(}\({\pvskey{TRUE}}\), \pvsid{convergence\_sequences}.\pvsid{limit}\pvsid{(}\pvsid{series}\pvsid{(}\(\lambda\) \(i\): \(S\)\pvsid{(}\(i\)\pvsid{)}`\(2\)\pvsid{)}\pvsid{)}\pvsid{)}
      \pvskey{ELSE} \pvsid{(}\({\pvskey{FALSE}}\), \(0\)\pvsid{)}
      \pvskey{ENDIF}\vspace*{\pvsdeclspacing}

  \pvsid{x\_converges?}\pvsid{(}\(S\), \(x\)\pvsid{)}: \pvsid{bool} \pvskey{=}
      \pvskey{IF} \pvsid{(}\(\forall\) \(i\): \(S\)\pvsid{(}\(i\)\pvsid{)}`\(1\)\pvsid{)} \(\wedge\) \pvsid{convergent}\pvsid{(}\(\lambda\) \(i\): \(S\)\pvsid{(}\(i\)\pvsid{)}`\(2\)\pvsid{)}
        \pvskey{THEN} \(x\)`\(1\) \(\wedge\) \(x\)`\(2\) \(=\) \pvsid{convergence\_sequences}.\pvsid{limit}\pvsid{(}\(\lambda\) \(i\): \(S\)\pvsid{(}\(i\)\pvsid{)}`\(2\)\pvsid{)}
      \pvskey{ELSE} \(\neg\) \(x\)`\(1\)
      \pvskey{ENDIF}\vspace*{\pvsdeclspacing}

  \(\xunderscorelimitonefn{S}\): \({\pvssubscript{\overline{\mathbb{R}}}{{\geq}0}}\) \pvskey{=}
      \pvskey{IF} \pvsid{(}\(\forall\) \(i\): \(S\)\pvsid{(}\(i\)\pvsid{)}`\(1\)\pvsid{)} \(\wedge\) \pvsid{convergent}\pvsid{(}\(\lambda\) \(i\): \(S\)\pvsid{(}\(i\)\pvsid{)}`\(2\)\pvsid{)}
        \pvskey{THEN} \pvsid{(}\({\pvskey{TRUE}}\), \pvsid{convergence\_sequences}.\pvsid{limit}\pvsid{(}\(\lambda\) \(i\): \(S\)\pvsid{(}\(i\)\pvsid{)}`\(2\)\pvsid{)}\pvsid{)}
      \pvskey{ELSE} \pvsid{(}\({\pvskey{FALSE}}\), \(0\)\pvsid{)}
      \pvskey{ENDIF}\vspace*{\pvsdeclspacing}

  \(\xunderscoreaddtwofn{x}{y}\): \({\pvssubscript{\overline{\mathbb{R}}}{{\geq}0}}\) \pvskey{=}
      \pvskey{IF} \(x\)`\(1\) \(\wedge\) \(y\)`\(1\)
        \pvskey{THEN} \pvsid{(}\({\pvskey{TRUE}}\), \(\opplustwofn{x`2}{y`2}\)\pvsid{)}
      \pvskey{ELSE} \pvsid{(}\({\pvskey{FALSE}}\), \(0\)\pvsid{)}
      \pvskey{ENDIF}\vspace*{\pvsdeclspacing}

  \(\xunderscoreaddtwofn{x}{c}\): \({\pvssubscript{\overline{\mathbb{R}}}{{\geq}0}}\) \pvskey{=}
      \pvskey{IF} \(x\)`\(1\) \pvskey{THEN} \pvsid{(}\({\pvskey{TRUE}}\), \(\opplustwofn{x`2}{c}\)\pvsid{)} \pvskey{ELSE} \pvsid{(}\({\pvskey{FALSE}}\), \(0\)\pvsid{)} \pvskey{ENDIF}\vspace*{\pvsdeclspacing}

  \(\xunderscoreeqtwofn{x}{y}\): \pvsid{bool} \pvskey{=} \pvsid{(}\(x\)`\(1\) \(=\) \(y\)`\(1\)\pvsid{)} \(\wedge\) \pvsid{(}\(x\)`\(1\) \(\Rightarrow\) \(x\)`\(2\) \(=\) \(y\)`\(2\)\pvsid{)}\vspace*{\pvsdeclspacing}

  \(\xunderscoreletwofn{x}{y}\): \pvsid{bool} \pvskey{=}
      \pvsid{(}\(x\)`\(1\) \(\wedge\) \(y\)`\(1\) \(\wedge\) \(x\)`\(2\) \(\leq\) \(y\)`\(2\)\pvsid{)} \(\vee\) \pvsid{(}\(\neg\) \(y\)`\(1\)\pvsid{)}\vspace*{\pvsdeclspacing}

  \(\xunderscorelttwofn{x}{y}\): \pvsid{bool} \pvskey{=}
      \pvsid{(}\(x\)`\(1\) \(\wedge\) \(y\)`\(1\) \(\wedge\) \(x\)`\(2\) \(<\) \(y\)`\(2\)\pvsid{)} \(\vee\) \pvsid{(}\(\neg\) \(y\)`\(1\)\pvsid{)}\vspace*{\pvsdeclspacing}

  \pvsid{x\_add\_commutative}: \pvskey{LEMMA} \pvsid{commutative?}\pvsid{(}\pvsid{x\_add}\pvsid{)}\vspace*{\pvsdeclspacing}

  \pvsid{x\_add\_associative}: \pvskey{LEMMA} \pvsid{associative?}\pvsid{(}\pvsid{x\_add}\pvsid{)}\vspace*{\pvsdeclspacing}

  \pvsid{x\_eq\_equivalence}: \pvskey{LEMMA} \pvsid{equivalence?}\pvsid{(}\pvsid{x\_eq}\pvsid{)}\vspace*{\pvsdeclspacing}

  \pvsid{x\_le\_preorder}: \pvskey{LEMMA} \pvsid{preorder?}\pvsid{(}\pvsid{x\_le}\pvsid{)}\vspace*{\pvsdeclspacing}

  \pvsid{x\_le\_antisymmetric}: \pvskey{LEMMA}
    \(\xunderscoreletwofn{x}{y}\) \(\wedge\) \(\xunderscoreletwofn{y}{x}\) \(\Rightarrow\) \(\xunderscoreeqtwofn{x}{y}\)\vspace*{\pvsdeclspacing}

  \pvsid{x\_sigma\_le}: \pvskey{LEMMA}
    \pvsid{(}\(\forall\) \(i\): \pvsid{low} \(\leq\) \(i\) \(\wedge\) \(i\) \(\leq\) \pvsid{high} \(\Rightarrow\) \(\xunderscoreletwofn{S\pvsid{(}i\pvsid{)}}{T\pvsid{(}i\pvsid{)}}\)\pvsid{)} \(\Rightarrow\)
     \(\xunderscoreletwofn{\xunderscoresigmathreefn{\pvsid{low}}{\pvsid{high}}{S}}{\xunderscoresigmathreefn{\pvsid{low}}{\pvsid{high}}{T}}\)\vspace*{\pvsdeclspacing}

  \pvsid{x\_sigma\_lt}: \pvskey{LEMMA}
    \pvsid{(}\(\forall\) \(i\): \pvsid{low} \(\leq\) \(i\) \(\wedge\) \(i\) \(\leq\) \pvsid{high} \(\Rightarrow\) \(\xunderscorelttwofn{S\pvsid{(}i\pvsid{)}}{T\pvsid{(}i\pvsid{)}}\)\pvsid{)} \(\wedge\) \pvsid{low} \(\leq\) \pvsid{high} \(\Rightarrow\)
     \(\xunderscorelttwofn{\xunderscoresigmathreefn{\pvsid{low}}{\pvsid{high}}{S}}{\xunderscoresigmathreefn{\pvsid{low}}{\pvsid{high}}{T}}\)\vspace*{\pvsdeclspacing}

  \pvsid{x\_sum\_le}: \pvskey{LEMMA}
    \pvsid{(}\(\forall\) \(i\): \(\xunderscoreletwofn{S\pvsid{(}i\pvsid{)}}{T\pvsid{(}i\pvsid{)}}\)\pvsid{)} \(\Rightarrow\) \(\xunderscoreletwofn{\xunderscoresumonefn{S}}{\xunderscoresumonefn{T}}\)\vspace*{\pvsdeclspacing}

  \pvsid{x\_sum\_eq}: \pvskey{LEMMA}
    \pvsid{(}\(\forall\) \(i\): \(\xunderscoreeqtwofn{S\pvsid{(}i\pvsid{)}}{T\pvsid{(}i\pvsid{)}}\)\pvsid{)} \(\Rightarrow\) \(\xunderscoreeqtwofn{\xunderscoresumonefn{S}}{\xunderscoresumonefn{T}}\)\vspace*{\pvsdeclspacing}

  \pvsid{x\_sum\_lt}: \pvskey{LEMMA}
    \pvsid{(}\(\forall\) \(i\): \(\xunderscorelttwofn{S\pvsid{(}i\pvsid{)}}{T\pvsid{(}i\pvsid{)}}\)\pvsid{)} \(\Rightarrow\) \(\xunderscorelttwofn{\xunderscoresumonefn{S}}{\xunderscoresumonefn{T}}\)\vspace*{\pvsdeclspacing}

  \pvsid{x\_inf\_le}: \pvskey{LEMMA}
    \pvsid{(}\(\forall\) \(i\): \(\xunderscoreletwofn{S\pvsid{(}i\pvsid{)}}{T\pvsid{(}i\pvsid{)}}\)\pvsid{)} \(\Rightarrow\) \(\xunderscoreletwofn{\xunderscoreinfonefn{S}}{\xunderscoreinfonefn{T}}\)\vspace*{\pvsdeclspacing}

  \pvsid{x\_le\_add}: \pvskey{LEMMA}
    \(\xunderscoreletwofn{x\sb{1}}{y\sb{1}}\) \(\wedge\) \(\xunderscoreletwofn{x\sb{2}}{y\sb{2}}\) \(\Rightarrow\)
     \(\xunderscoreletwofn{\xunderscoreaddtwofn{x\sb{1}}{x\sb{2}}}{\xunderscoreaddtwofn{y\sb{1}}{y\sb{2}}}\)\vspace*{\pvsdeclspacing}

  \pvsid{x\_add\_sum}: \pvskey{LEMMA}
    \(\xunderscoreeqtwofn{\xunderscoreaddtwofn{\xunderscoresumonefn{S}}{\xunderscoresumonefn{T}}}{\xunderscoresumonefn{\lambda i: \xunderscoreaddtwofn{S\pvsid{(}i\pvsid{)}}{T\pvsid{(}i\pvsid{)}}}}\)\vspace*{\pvsdeclspacing}

  \pvsid{epsilon\_x\_le}: \pvskey{LEMMA}
    \(\xunderscoreletwofn{x}{y}\) \(\Leftrightarrow\) \pvsid{(}\(\forall\) \(\varepsilon\): \(\xunderscoreletwofn{x}{\xunderscoreaddtwofn{y}{\varepsilon}}\)\pvsid{)}\vspace*{\pvsdeclspacing}

  \pvskey{IMPORTING} \pvsid{double\_index}\({\pvsbracketl}\)\({\pvssubscript{\overline{\mathbb{R}}}{{\geq}0}}\)\({\pvsbracketr}\), \pvsid{double\_nn\_sequence}

  \(U\): \pvskey{VAR} \({\pvsbracketl}\)\({\pvsbracketl}\)\(\mathbb{N}\), \(\mathbb{N}\)\({\pvsbracketr}\) \(\rightarrow\) \({\pvssubscript{\overline{\mathbb{R}}}{{\geq}0}}\)\({\pvsbracketr}\)\vspace*{\pvsdeclspacing}

  \pvsid{double\_x\_sum}: \pvskey{LEMMA}
    \(\xunderscoreeqtwofn{{\zbox{\({\xunderscoresumonefn{{\zbox{\({\pvsid{single\_index}\pvsid{(}U\pvsid{)}}\)}}}}\)}}}{{\zbox{\({\xunderscoresumonefn{{\zbox{\({\lambda i:}\)} \hbox{\({\xunderscoresumonefn{{\hbox{\({\lambda j: U\pvsid{(}i, j\pvsid{)}}\)}}}}\)}}}}\)}}}\)\vspace*{\pvsdeclspacing}

  \pvsid{double\_x\_sum\_eq}: \pvskey{LEMMA}
    \(\xunderscoreeqtwofn{{\zbox{\({\xunderscoresumonefn{{\zbox{\({\lambda i: \xunderscoresumonefn{{\zbox{\({\lambda j: U\pvsid{(}i, j\pvsid{)}}\)}}}}\)}}}}\)}}}{{\zbox{\({\xunderscoresumonefn{{\zbox{\({\lambda j:}\)} \zbox{\({\xunderscoresumonefn{{\zbox{\({\lambda}\)} \zbox{\({i:}\)} \hbox{\({U\pvsid{(}i, j\pvsid{)}}\)}}}}\)}}}}\)}}}\)\vspace*{\pvsdeclspacing}

 \pvskey{END} \({\pvssubscript{\overline{\mathbb{R}}}{{\geq}0}}\)\end{alltt}

\newpage
\section{Library code\_product}
% The following substitutions are from the file:
%   /home/david/PVS42/pvs-tex.sub
\def\xunderscorelttwofn#1#2{{#1 < #2}}% How to print function x_lt with arity (2)
\def\xunderscoreletwofn#1#2{{#1 \leq #2}}% How to print function x_le with arity (2)
\def\xunderscoreeqtwofn#1#2{{#1 = #2}}% How to print function x_eq with arity (2)
\def\xunderscoreaddtwofn#1#2{{#1 + #2}}% How to print function x_add with arity (2)
\def\xunderscorelimitonefn#1{{\pvsid{limit}(#1)}}% How to print function x_limit with arity (1)
\def\xunderscoresumonefn#1{{\sum #1}}% How to print function x_sum with arity (1)
\def\xunderscoresigmathreefn#1#2#3{{\sum_{#1}^{#2} #3}}% How to print function x_sigma with arity (3)
\def\xunderscoresuponefn#1{{\pvsid{sup}(#1)}}% How to print function x_sup with arity (1)
\def\xunderscoreinfonefn#1{{\pvsid{inf}(#1)}}% How to print function x_inf with arity (1)
\def\Etwofn#1#2{{\mathbb{E}(#1~|~#2)}}% How to print function E with arity (2)
\def\Eonefn#1{{\mathbb{E}(#1)}}% How to print function E with arity (1)
\def\Ptwofn#1#2{{\mathbb{P}(#1~|~#2)}}% How to print function P with arity (2)
\def\Ponefn#1{{\mathbb{P}(#1)}}% How to print function P with arity (1)
\def\xtwofn#1#2{{#1\times#2}}% How to print function x with arity (2)
\def\asttwofn#1#2{{#1\ast#2}}% How to print function ast with arity (2)
\def\astonefn#1{{#1\ast}}% How to print function ast with arity (1)
\def\dottwofn#1#2{{#1\bullet#2}}% How to print function dot with arity (2)
\def\integralthreefn#1#2#3{{\int_{#1}^{#2} #3}}% How to print function integral with arity (3)
\def\integraltwofn#1#2{{\int_{#1} #2}}% How to print function integral with arity (2)
\def\integralonefn#1{{\int#1}}% How to print function integral with arity (1)
\def\normonefn#1{{\left||{#1}\right||}}% How to print function norm with arity (1)
\def\phionefn#1{{\pvssubscript{\phi}{#1}}}% How to print function phi with arity (1)
\def\infunderscoreclosedonefn#1{{\left(-\infty,~#1\right]}}% How to print function inf_closed with arity (1)
\def\closedunderscoreinfonefn#1{{\left[#1,~\infty\right)}}% How to print function closed_inf with arity (1)
\def\infunderscoreopenonefn#1{(-\infty,~#1)}% How to print function inf_open with arity (1)
\def\openunderscoreinfonefn#1{(#1,~\infty)}% How to print function open_inf with arity (1)
\def\closedtwofn#1#2{{\left[#1,~#2\right]}}% How to print function closed with arity (2)
\def\opentwofn#1#2{(#1,~#2)}% How to print function open with arity (2)
\def\sigmathreefn#1#2#3{{\sum_{#1}^{#2} #3}}% How to print function sigma with arity (3)
\def\sigmatwofn#1#2{{\sum_{#1} {#2}}}% How to print function sigma with arity (2)
\def\ceilingonefn#1{{\lceil{#1}\rceil}}% How to print function ceiling with arity (1)
\def\flooronefn#1{{\lfloor{#1}\rfloor}}% How to print function floor with arity (1)
\def\absonefn#1{{\left|{#1}\right|}}% How to print function abs with arity (1)
\def\roottwofn#1#2{{\sqrt[#2]{#1}}}% How to print function root with arity (2)
\def\sqrtonefn#1{{\sqrt{#1}}}% How to print function sqrt with arity (1)
\def\sqonefn#1{{\pvssuperscript{#1}{2}}}% How to print function sq with arity (1)
\def\expttwofn#1#2{{\pvssuperscript{#1}{#2}}}% How to print function expt with arity (2)
\def\opcarettwofn#1#2{{\pvssuperscript{#1}{#2}}}% How to print function ^ with arity (2)
\def\setsotherremovetwofn#1#2{{(#2 \setminus \{#1\})}}% How to print function sets.remove with arity (2)
\def\setsotheraddtwofn#1#2{{(#2 \cup \{#1\})}}% How to print function sets.add with arity (2)
\def\setsotherdifferencetwofn#1#2{{(#1 \setminus #2)}}% How to print function sets.difference with arity (2)
\def\setsothercomplementonefn#1{{\overline{#1}}}% How to print function sets.complement with arity (1)
\def\setsotherintersectiontwofn#1#2{{(#1 \cap #2)}}% How to print function sets.intersection with arity (2)
\def\setsotheruniontwofn#1#2{{(#1 \cup #2)}}% How to print function sets.union with arity (2)
\def\setsotherstrictunderscoresubsetothertwofn#1#2{{(#1 \subset #2)}}% How to print function sets.strict_subset? with arity (2)
\def\setsothersubsetothertwofn#1#2{{(#1 \subseteq #2)}}% How to print function sets.subset? with arity (2)
\def\setsothermembertwofn#1#2{{(#1 \in #2)}}% How to print function sets.member with arity (2)
\def\opohtwofn#1#2{{#1 \circ #2}}% How to print function O with arity (2)
\def\opdividetwofn#1#2{{\frac{#1}{#2}}}% How to print function / with arity (2)
\def\optimestwofn#1#2{{#1\times#2}}% How to print function * with arity (2)
\def\opdifferenceonefn#1{{-#1}}% How to print function - with arity (1)
\def\opdifferencetwofn#1#2{{#1-#2}}% How to print function - with arity (2)
\def\opplustwofn#1#2{{#1+#2}}% How to print function + with arity (2)
\begin{alltt}
\pvsid{code\_product}: \pvskey{THEORY}
 \pvskey{BEGIN}

  \pvskey{IMPORTING} \pvsid{reals}@\pvsid{sqrt}

  \(i\), \(j\), \(n\), \(m\): \pvskey{VAR} \(\mathbb{N}\)\vspace*{\pvsdeclspacing}

  \pvsid{double\_index\_n}\pvsid{(}\(i\), \(j\)\pvsid{)}: \(\mathbb{N}\) \pvskey{=}
      \(\opplustwofn{\opdividetwofn{\pvsid{(}\optimestwofn{\pvsid{(}\opplustwofn{i}{j}\pvsid{)}}{\pvsid{(}\opplustwofn{\opplustwofn{1}{i}}{j}\pvsid{)}}\pvsid{)}}{2}}{j}\)\vspace*{\pvsdeclspacing}

  \pvsid{double\_index\_triangle}\pvsid{(}\(n\)\pvsid{)}: \(\mathbb{N}\) \pvskey{=}
      \pvsid{singleton\_elt}\({\pvsbracketl}\)\(\mathbb{N}\)\({\pvsbracketr}\)
          \pvsid{(}\{\(i\) |
                \(\opdividetwofn{\pvsid{(}\optimestwofn{i}{\pvsid{(}\opplustwofn{i}{1}\pvsid{)}}\pvsid{)}}{2}\) \(\leq\) \(n\) \(\wedge\)
                 \(n\) \(<\) \(\opdividetwofn{\pvsid{(}\optimestwofn{\pvsid{(}\opplustwofn{i}{1}\pvsid{)}}{\pvsid{(}\opplustwofn{i}{2}\pvsid{)}}\pvsid{)}}{2}\)\}\pvsid{)}\vspace*{\pvsdeclspacing}

  \pvsid{double\_index\_triangle\_def}: \pvskey{LEMMA}
    \(\opdividetwofn{\pvsid{(}\optimestwofn{\pvsid{double\_index\_triangle}\pvsid{(}n\pvsid{)}}{\pvsid{(}\opplustwofn{\pvsid{double\_index\_triangle}\pvsid{(}n\pvsid{)}}{1}\pvsid{)}}\pvsid{)}}{2}\) \(\leq\) \(n\) \(\wedge\)
     \(n\) \(<\)
      \(\opdividetwofn{\pvsid{(}\optimestwofn{\pvsid{(}\opplustwofn{\pvsid{double\_index\_triangle}\pvsid{(}n\pvsid{)}}{1}\pvsid{)}}{\pvsid{(}\opplustwofn{\pvsid{double\_index\_triangle}\pvsid{(}n\pvsid{)}}{2}\pvsid{)}}\pvsid{)}}{2}\)\vspace*{\pvsdeclspacing}

  \pvsid{double\_index\_triangle\_increasing}: \pvskey{LEMMA}
    \(\forall\) \(i\), \(j\):
      \(i\) \(\leq\) \(j\) \(\Rightarrow\)
       \pvsid{double\_index\_triangle}\pvsid{(}\(i\)\pvsid{)} \(\leq\) \pvsid{double\_index\_triangle}\pvsid{(}\(j\)\pvsid{)}\vspace*{\pvsdeclspacing}

  \pvsid{double\_index\_triangle\_bound}: \pvskey{LEMMA} \pvsid{double\_index\_triangle}\pvsid{(}\(n\)\pvsid{)} \(\leq\) \(n\)\vspace*{\pvsdeclspacing}

  \pvsid{double\_index\_j}\pvsid{(}\(n\)\pvsid{)}: \(\mathbb{N}\) \pvskey{=}
      \(\opdifferencetwofn{n}{\opdividetwofn{\pvsid{(}\optimestwofn{\pvsid{double\_index\_triangle}\pvsid{(}n\pvsid{)}}{\pvsid{(}\opplustwofn{\pvsid{double\_index\_triangle}\pvsid{(}n\pvsid{)}}{1}\pvsid{)}}\pvsid{)}}{2}}\)\vspace*{\pvsdeclspacing}

  \pvsid{double\_index\_j\_bound}: \pvskey{LEMMA}
    \pvsid{double\_index\_j}\pvsid{(}\(n\)\pvsid{)} \(\leq\) \pvsid{double\_index\_triangle}\pvsid{(}\(n\)\pvsid{)}\vspace*{\pvsdeclspacing}

  \pvsid{double\_index\_i}\pvsid{(}\(n\)\pvsid{)}: \(\mathbb{N}\) \pvskey{=}
      \(\opdifferencetwofn{\pvsid{double\_index\_triangle}\pvsid{(}n\pvsid{)}}{\pvsid{double\_index\_j}\pvsid{(}n\pvsid{)}}\)\vspace*{\pvsdeclspacing}

  \pvsid{double\_index\_i\_bound}: \pvskey{LEMMA}
    \pvsid{double\_index\_i}\pvsid{(}\(n\)\pvsid{)} \(\leq\) \pvsid{double\_index\_triangle}\pvsid{(}\(n\)\pvsid{)}\vspace*{\pvsdeclspacing}

  \pvsid{double\_index\_n\_increasing}: \pvskey{LEMMA}
    \(\forall\) \(n\), \(m\):
      \(n\) \(<\) \(m\) \(\Rightarrow\) \pvsid{double\_index\_n}\pvsid{(}\(n\), \(j\)\pvsid{)} \(<\) \pvsid{double\_index\_n}\pvsid{(}\(m\), \(j\)\pvsid{)}\vspace*{\pvsdeclspacing}

  \pvsid{double\_index\_n\_ij}: \pvskey{LEMMA}
    \pvsid{double\_index\_n}\pvsid{(}\pvsid{double\_index\_i}\pvsid{(}\(n\)\pvsid{)}, \pvsid{double\_index\_j}\pvsid{(}\(n\)\pvsid{)}\pvsid{)} \(=\) \(n\)\vspace*{\pvsdeclspacing}

  \pvsid{double\_index\_ij\_n}: \pvskey{LEMMA}
    \pvsid{double\_index\_i}\pvsid{(}\pvsid{double\_index\_n}\pvsid{(}\(i\), \(j\)\pvsid{)}\pvsid{)} \(=\) \(i\) \(\wedge\)
     \pvsid{double\_index\_j}\pvsid{(}\pvsid{double\_index\_n}\pvsid{(}\(i\), \(j\)\pvsid{)}\pvsid{)} \(=\) \(j\)\vspace*{\pvsdeclspacing}

  \pvsid{double\_index\_n\_bijective}: \pvskey{LEMMA}
    \pvsid{bijective?}\({\pvsbracketl}\)\({\pvsbracketl}\)\(\mathbb{N}\), \(\mathbb{N}\)\({\pvsbracketr}\), \(\mathbb{N}\)\({\pvsbracketr}\)\pvsid{(}\pvsid{double\_index\_n}\pvsid{)}\vspace*{\pvsdeclspacing}

 \pvskey{END} \pvsid{code\_product}\end{alltt}

\newpage
\section{Library double\_index}
\input{double_index.tex}
\newpage
\section{Library double\_nn\_sequence}
% The following substitutions are from the file:
%   /home/david/PVS42/pvs-tex.sub
\def\xunderscorelttwofn#1#2{{#1 < #2}}% How to print function x_lt with arity (2)
\def\xunderscoreletwofn#1#2{{#1 \leq #2}}% How to print function x_le with arity (2)
\def\xunderscoreeqtwofn#1#2{{#1 = #2}}% How to print function x_eq with arity (2)
\def\xunderscoreaddtwofn#1#2{{#1 + #2}}% How to print function x_add with arity (2)
\def\xunderscorelimitonefn#1{{\pvsid{limit}(#1)}}% How to print function x_limit with arity (1)
\def\xunderscoresumonefn#1{{\sum #1}}% How to print function x_sum with arity (1)
\def\xunderscoresigmathreefn#1#2#3{{\sum_{#1}^{#2} #3}}% How to print function x_sigma with arity (3)
\def\xunderscoresuponefn#1{{\pvsid{sup}(#1)}}% How to print function x_sup with arity (1)
\def\xunderscoreinfonefn#1{{\pvsid{inf}(#1)}}% How to print function x_inf with arity (1)
\def\Etwofn#1#2{{\mathbb{E}(#1~|~#2)}}% How to print function E with arity (2)
\def\Eonefn#1{{\mathbb{E}(#1)}}% How to print function E with arity (1)
\def\Ptwofn#1#2{{\mathbb{P}(#1~|~#2)}}% How to print function P with arity (2)
\def\Ponefn#1{{\mathbb{P}(#1)}}% How to print function P with arity (1)
\def\xtwofn#1#2{{#1\times#2}}% How to print function x with arity (2)
\def\asttwofn#1#2{{#1\ast#2}}% How to print function ast with arity (2)
\def\astonefn#1{{#1\ast}}% How to print function ast with arity (1)
\def\dottwofn#1#2{{#1\bullet#2}}% How to print function dot with arity (2)
\def\integralthreefn#1#2#3{{\int_{#1}^{#2} #3}}% How to print function integral with arity (3)
\def\integraltwofn#1#2{{\int_{#1} #2}}% How to print function integral with arity (2)
\def\integralonefn#1{{\int#1}}% How to print function integral with arity (1)
\def\normonefn#1{{\left||{#1}\right||}}% How to print function norm with arity (1)
\def\phionefn#1{{\pvssubscript{\phi}{#1}}}% How to print function phi with arity (1)
\def\infunderscoreclosedonefn#1{{\left(-\infty,~#1\right]}}% How to print function inf_closed with arity (1)
\def\closedunderscoreinfonefn#1{{\left[#1,~\infty\right)}}% How to print function closed_inf with arity (1)
\def\infunderscoreopenonefn#1{(-\infty,~#1)}% How to print function inf_open with arity (1)
\def\openunderscoreinfonefn#1{(#1,~\infty)}% How to print function open_inf with arity (1)
\def\closedtwofn#1#2{{\left[#1,~#2\right]}}% How to print function closed with arity (2)
\def\opentwofn#1#2{(#1,~#2)}% How to print function open with arity (2)
\def\sigmathreefn#1#2#3{{\sum_{#1}^{#2} #3}}% How to print function sigma with arity (3)
\def\sigmatwofn#1#2{{\sum_{#1} {#2}}}% How to print function sigma with arity (2)
\def\ceilingonefn#1{{\lceil{#1}\rceil}}% How to print function ceiling with arity (1)
\def\flooronefn#1{{\lfloor{#1}\rfloor}}% How to print function floor with arity (1)
\def\absonefn#1{{\left|{#1}\right|}}% How to print function abs with arity (1)
\def\roottwofn#1#2{{\sqrt[#2]{#1}}}% How to print function root with arity (2)
\def\sqrtonefn#1{{\sqrt{#1}}}% How to print function sqrt with arity (1)
\def\sqonefn#1{{\pvssuperscript{#1}{2}}}% How to print function sq with arity (1)
\def\expttwofn#1#2{{\pvssuperscript{#1}{#2}}}% How to print function expt with arity (2)
\def\opcarettwofn#1#2{{\pvssuperscript{#1}{#2}}}% How to print function ^ with arity (2)
\def\setsotherremovetwofn#1#2{{(#2 \setminus \{#1\})}}% How to print function sets.remove with arity (2)
\def\setsotheraddtwofn#1#2{{(#2 \cup \{#1\})}}% How to print function sets.add with arity (2)
\def\setsotherdifferencetwofn#1#2{{(#1 \setminus #2)}}% How to print function sets.difference with arity (2)
\def\setsothercomplementonefn#1{{\overline{#1}}}% How to print function sets.complement with arity (1)
\def\setsotherintersectiontwofn#1#2{{(#1 \cap #2)}}% How to print function sets.intersection with arity (2)
\def\setsotheruniontwofn#1#2{{(#1 \cup #2)}}% How to print function sets.union with arity (2)
\def\setsotherstrictunderscoresubsetothertwofn#1#2{{(#1 \subset #2)}}% How to print function sets.strict_subset? with arity (2)
\def\setsothersubsetothertwofn#1#2{{(#1 \subseteq #2)}}% How to print function sets.subset? with arity (2)
\def\setsothermembertwofn#1#2{{(#1 \in #2)}}% How to print function sets.member with arity (2)
\def\opohtwofn#1#2{{#1 \circ #2}}% How to print function O with arity (2)
\def\opdividetwofn#1#2{{\frac{#1}{#2}}}% How to print function / with arity (2)
\def\optimestwofn#1#2{{#1\times#2}}% How to print function * with arity (2)
\def\opdifferenceonefn#1{{-#1}}% How to print function - with arity (1)
\def\opdifferencetwofn#1#2{{#1-#2}}% How to print function - with arity (2)
\def\opplustwofn#1#2{{#1+#2}}% How to print function + with arity (2)
\begin{alltt}
\pvsid{double\_nn\_sequence}: \pvskey{THEORY}
 \pvskey{BEGIN}

  \pvskey{IMPORTING} \pvsid{analysis}@\pvsid{real\_fun\_supinf}\({\pvsbracketl}\)\({\pvssubscript{\mathbb{R}}{{\geq}0}}\)\({\pvsbracketr}\), \pvsid{double\_index}\({\pvsbracketl}\)\({\pvssubscript{\mathbb{R}}{{\geq}0}}\)\({\pvsbracketr}\), \pvsid{sigma\_set}@\pvsid{series\_aux},
            \pvsid{sigma\_set}@\pvsid{absconv\_series}

  \(u\): \pvskey{VAR} \({\pvsbracketl}\)\({\pvsbracketl}\)\(\mathbb{N}\), \(\mathbb{N}\)\({\pvsbracketr}\) \(\rightarrow\) \({\pvssubscript{\mathbb{R}}{{\geq}0}}\)\({\pvsbracketr}\)\vspace*{\pvsdeclspacing}

  \(v\): \pvskey{VAR} \({\pvsbracketl}\)\(\mathbb{N}\) \(\rightarrow\) \({\pvssubscript{\mathbb{R}}{{\geq}0}}\)\({\pvsbracketr}\)\vspace*{\pvsdeclspacing}

  \(l\): \pvskey{VAR} \({\pvssubscript{\mathbb{R}}{{\geq}0}}\)\vspace*{\pvsdeclspacing}

  \(i\), \(j\), \(n\): \pvskey{VAR} \(\mathbb{N}\)\vspace*{\pvsdeclspacing}

  \(z\): \pvskey{VAR} \(\mathbb{R}\)\vspace*{\pvsdeclspacing}

  \pvsid{nn\_series\_increasing}: \pvskey{LEMMA} \pvsid{increasing?}\pvsid{(}\pvsid{series}\pvsid{(}\(v\)\pvsid{)}\pvsid{)}\vspace*{\pvsdeclspacing}

  \pvsid{nn\_index\_scaf1}: \pvskey{LEMMA}
    \pvsid{series}\pvsid{(}\pvsid{single\_index}\pvsid{(}\(u\)\pvsid{)}\pvsid{)}\pvsid{(}\pvsid{double\_index\_n}\pvsid{(}\(0\), \(n\)\pvsid{)}\pvsid{)} \(=\)
     \(\sigmathreefn{0}{n}{\lambda i: \sigmathreefn{0}{\opdifferencetwofn{n}{i}}{\lambda j: u\pvsid{(}i, j\pvsid{)}}}\)\vspace*{\pvsdeclspacing}

  \pvsid{nn\_double\_index\_incr}: \pvskey{LEMMA}
    \pvsid{double\_index\_n}\pvsid{(}\(i\), \(j\)\pvsid{)} \(=\) \(\opdifferencetwofn{\pvsid{double\_index\_n}\pvsid{(}0, \opplustwofn{i}{j}\pvsid{)}}{i}\)\vspace*{\pvsdeclspacing}

  \pvsid{nn\_index\_scaf2}: \pvskey{LEMMA}
    \(\sigmathreefn{0}{n}{\lambda i: \sigmathreefn{0}{n}{\lambda j: u\pvsid{(}i, j\pvsid{)}}}\) \(\leq\)
     \pvsid{series}\pvsid{(}\pvsid{single\_index}\pvsid{(}\(u\)\pvsid{)}\pvsid{)}\pvsid{(}\pvsid{double\_index\_n}\pvsid{(}\(0\), \(\optimestwofn{2}{n}\)\pvsid{)}\pvsid{)}\vspace*{\pvsdeclspacing}

  \pvsid{nn\_index\_scaf3}: \pvskey{LEMMA}
    \(\exists\) \(n\):
      \(\sigmathreefn{0}{i}{\lambda i: \sigmathreefn{0}{j}{\lambda j: u\pvsid{(}i, j\pvsid{)}}}\) \(\leq\)
       \pvsid{series}\pvsid{(}\pvsid{single\_index}\pvsid{(}\(u\)\pvsid{)}\pvsid{)}\pvsid{(}\(n\)\pvsid{)}\vspace*{\pvsdeclspacing}

  \pvsid{nn\_index\_scaf4}: \pvskey{LEMMA}
    \(\exists\) \(i\), \(j\):
      \pvsid{series}\pvsid{(}\pvsid{single\_index}\pvsid{(}\(u\)\pvsid{)}\pvsid{)}\pvsid{(}\(n\)\pvsid{)} \(\leq\)
       \(\sigmathreefn{0}{i}{\lambda i: \sigmathreefn{0}{j}{\lambda j: u\pvsid{(}i, j\pvsid{)}}}\)\vspace*{\pvsdeclspacing}

  \pvsid{nn\_convegent\_bounded}: \pvskey{LEMMA}
    \pvsid{convergent}\pvsid{(}\pvsid{series}\pvsid{(}\(v\)\pvsid{)}\pvsid{)} \(\equiv\) \pvsid{bounded\_above?}\pvsid{(}\pvsid{Im}\pvsid{(}\pvsid{series}\pvsid{(}\(v\)\pvsid{)}\pvsid{)}\pvsid{)}\vspace*{\pvsdeclspacing}

  \pvsid{nn\_limit\_lub}: \pvskey{LEMMA}
    \pvsid{convergent}\pvsid{(}\pvsid{series}\pvsid{(}\(v\)\pvsid{)}\pvsid{)} \(\Rightarrow\)
     \pvsid{limit}\pvsid{(}\pvsid{series}\pvsid{(}\(v\)\pvsid{)}\pvsid{)} \(=\) \pvsid{lub}\pvsid{(}\pvsid{Im}\pvsid{(}\pvsid{series}\pvsid{(}\(v\)\pvsid{)}\pvsid{)}\pvsid{)}\vspace*{\pvsdeclspacing}

  \pvsid{nn\_convergence\_least\_upper\_bound}: \pvskey{LEMMA}
    \pvsid{increasing?}\pvsid{(}\(v\)\pvsid{)} \(\Rightarrow\)
     \pvsid{(}\pvsid{convergence}\pvsid{(}\(v\), \(l\)\pvsid{)} \(\equiv\) \pvsid{least\_upper\_bound?}\pvsid{(}\(l\), \pvsid{Im}\pvsid{(}\(v\)\pvsid{)}\pvsid{)}\pvsid{)}\vspace*{\pvsdeclspacing}

  \pvsid{double\_series}: \pvskey{LEMMA}
    \pvsid{series}\pvsid{(}\(\lambda\) \(i\): \pvsid{series}\pvsid{(}\(\lambda\) \(j\): \(u\)\pvsid{(}\(i\), \(j\)\pvsid{)}\pvsid{)}\pvsid{(}\(j\)\pvsid{)}\pvsid{)}\pvsid{(}\(i\)\pvsid{)} \(=\)
     \pvsid{series}\pvsid{(}\(\lambda\) \(j\): \pvsid{series}\pvsid{(}\(\lambda\) \(i\): \(u\)\pvsid{(}\(i\), \(j\)\pvsid{)}\pvsid{)}\pvsid{(}\(i\)\pvsid{)}\pvsid{)}\pvsid{(}\(j\)\pvsid{)}\vspace*{\pvsdeclspacing}

  \pvsid{double\_subseq\_convergent}: \pvskey{LEMMA}
    \pvsid{convergent}\pvsid{(}\pvsid{series}\pvsid{(}\pvsid{single\_index}\pvsid{(}\(u\)\pvsid{)}\pvsid{)}\pvsid{)} \(\Rightarrow\)
     \pvsid{convergent}\pvsid{(}\pvsid{series}\pvsid{(}\(\lambda\) \(j\): \(u\)\pvsid{(}\(i\), \(j\)\pvsid{)}\pvsid{)}\pvsid{)}\vspace*{\pvsdeclspacing}

  \pvsid{double\_subseq\_bounded}: \pvskey{LEMMA}
    \pvsid{bounded\_above?}\pvsid{(}\pvsid{Im}\pvsid{(}\pvsid{series}\pvsid{(}\pvsid{single\_index}\pvsid{(}\(u\)\pvsid{)}\pvsid{)}\pvsid{)}\pvsid{)} \(\Rightarrow\)
     \pvsid{bounded\_above?}\pvsid{(}\pvsid{Im}\pvsid{(}\pvsid{series}\pvsid{(}\(\lambda\) \(j\): \(u\)\pvsid{(}\(i\), \(j\)\pvsid{)}\pvsid{)}\pvsid{)}\pvsid{)}\vspace*{\pvsdeclspacing}

  \pvsid{series\_limit\_def}: \pvskey{LEMMA}
    \pvsid{(}\(\forall\) \(i\): \pvsid{convergent}\pvsid{(}\(\lambda\) \(j\): \(u\)\pvsid{(}\(i\), \(j\)\pvsid{)}\pvsid{)}\pvsid{)} \(\Rightarrow\)
     \pvsid{series}\pvsid{(}\(\lambda\) \(i\): \pvsid{limit}\pvsid{(}\(\lambda\) \(j\): \(u\)\pvsid{(}\(i\), \(j\)\pvsid{)}\pvsid{)}\pvsid{)}\pvsid{(}\(i\)\pvsid{)} \(=\)
      \pvsid{limit}\pvsid{(}\(\lambda\) \(j\): \pvsid{series}\pvsid{(}\(\lambda\) \(i\): \(u\)\pvsid{(}\(i\), \(j\)\pvsid{)}\pvsid{)}\pvsid{(}\(i\)\pvsid{)}\pvsid{)}\vspace*{\pvsdeclspacing}

  \pvsid{double\_approx}: \pvskey{LEMMA}
    \pvsid{(}\(\forall\) \(i\): \pvsid{convergent}\pvsid{(}\pvsid{series}\pvsid{(}\(\lambda\) \(j\): \(u\)\pvsid{(}\(i\), \(j\)\pvsid{)}\pvsid{)}\pvsid{)}\pvsid{)} \(\Rightarrow\)
     \pvsid{series}\pvsid{(}\(\lambda\) \(i\): \pvsid{limit}\pvsid{(}\pvsid{series}\pvsid{(}\(\lambda\) \(j\): \(u\)\pvsid{(}\(i\), \(j\)\pvsid{)}\pvsid{)}\pvsid{)}\pvsid{)}\pvsid{(}\(i\)\pvsid{)} \(=\)
      \pvsid{limit}\pvsid{(}\pvsid{series}\pvsid{(}\(\lambda\) \(j\): \pvsid{series}\pvsid{(}\(\lambda\) \(i\): \(u\)\pvsid{(}\(i\), \(j\)\pvsid{)}\pvsid{)}\pvsid{(}\(i\)\pvsid{)}\pvsid{)}\pvsid{)}\vspace*{\pvsdeclspacing}

  \pvsid{double\_approx1}: \pvskey{LEMMA}
    \pvsid{convergence}\pvsid{(}\pvsid{series}\pvsid{(}\pvsid{single\_index}\pvsid{(}\(u\)\pvsid{)}\pvsid{)}, \(l\)\pvsid{)} \(\equiv\)
     \pvsid{least\_upper\_bound?}\pvsid{(}\(l\),
                         \{\(z\) |
                             \(\exists\) \(i\), \(j\):
                               \pvsid{series}\pvsid{(}\(\lambda\) \(j\): \pvsid{series}\pvsid{(}\(\lambda\) \(i\): \(u\)\pvsid{(}\(i\), \(j\)\pvsid{)}\pvsid{)}\pvsid{(}\(i\)\pvsid{)}\pvsid{)}\pvsid{(}\(j\)\pvsid{)} \(=\)
                                \(z\)\}\pvsid{)}\vspace*{\pvsdeclspacing}

  \pvsid{double\_approx2}: \pvskey{LEMMA}
    \pvsid{(}\(\forall\) \(i\): \pvsid{convergent}\pvsid{(}\pvsid{series}\pvsid{(}\(\lambda\) \(j\): \(u\)\pvsid{(}\(i\), \(j\)\pvsid{)}\pvsid{)}\pvsid{)}\pvsid{)} \(\Rightarrow\)
     \pvsid{(}\pvsid{convergence}\pvsid{(}\pvsid{series}\pvsid{(}\(\lambda\) \(i\): \pvsid{limit}\pvsid{(}\pvsid{series}\pvsid{(}\(\lambda\) \(j\): \(u\)\pvsid{(}\(i\), \(j\)\pvsid{)}\pvsid{)}\pvsid{)}\pvsid{)}, \(l\)\pvsid{)} \(\equiv\)
        \pvsid{least\_upper\_bound?}\pvsid{(}\(l\),
                            \{\(z\) |
                                \(\exists\) \(i\), \(j\):
                                  \pvsid{series}\pvsid{(}\(\lambda\) \(j\): \pvsid{series}\pvsid{(}\(\lambda\) \(i\): \(u\)\pvsid{(}\(i\), \(j\)\pvsid{)}\pvsid{)}\pvsid{(}\(i\)\pvsid{)}\pvsid{)}\pvsid{(}\(j\)\pvsid{)} \(=\)
                                   \(z\)\}\pvsid{)}\pvsid{)}\vspace*{\pvsdeclspacing}

  \pvsid{double\_left\_convergence}: \pvskey{LEMMA}
    \pvsid{convergence}\pvsid{(}\pvsid{series}\pvsid{(}\pvsid{single\_index}\pvsid{(}\(u\)\pvsid{)}\pvsid{)}, \(l\)\pvsid{)} \(\equiv\)
     \pvsid{(}\(\forall\) \(i\): \pvsid{convergent}\pvsid{(}\pvsid{series}\pvsid{(}\(\lambda\) \(j\): \(u\)\pvsid{(}\(i\), \(j\)\pvsid{)}\pvsid{)}\pvsid{)}\pvsid{)} \(\wedge\)
      \pvsid{convergence}\pvsid{(}\pvsid{series}\pvsid{(}\(\lambda\) \(i\): \pvsid{limit}\pvsid{(}\pvsid{series}\pvsid{(}\(\lambda\) \(j\): \(u\)\pvsid{(}\(i\), \(j\)\pvsid{)}\pvsid{)}\pvsid{)}\pvsid{)},
                   \(l\)\pvsid{)}\vspace*{\pvsdeclspacing}

  \pvsid{double\_left\_convergent}: \pvskey{LEMMA}
    \pvsid{convergent}\pvsid{(}\pvsid{series}\pvsid{(}\pvsid{single\_index}\pvsid{(}\(u\)\pvsid{)}\pvsid{)}\pvsid{)} \(\equiv\)
     \pvsid{(}\(\forall\) \(i\): \pvsid{convergent}\pvsid{(}\pvsid{series}\pvsid{(}\(\lambda\) \(j\): \(u\)\pvsid{(}\(i\), \(j\)\pvsid{)}\pvsid{)}\pvsid{)}\pvsid{)} \(\wedge\)
      \pvsid{convergent}\pvsid{(}\pvsid{series}\pvsid{(}\(\lambda\) \(i\): \pvsid{limit}\pvsid{(}\pvsid{series}\pvsid{(}\(\lambda\) \(j\): \(u\)\pvsid{(}\(i\), \(j\)\pvsid{)}\pvsid{)}\pvsid{)}\pvsid{)}\pvsid{)}\vspace*{\pvsdeclspacing}

  \pvsid{double\_left\_limit}: \pvskey{LEMMA}
    \pvsid{convergent}\pvsid{(}\pvsid{series}\pvsid{(}\pvsid{single\_index}\pvsid{(}\(u\)\pvsid{)}\pvsid{)}\pvsid{)} \(\Rightarrow\)
     \pvsid{limit}\pvsid{(}\pvsid{series}\pvsid{(}\pvsid{single\_index}\pvsid{(}\(u\)\pvsid{)}\pvsid{)}\pvsid{)} \(=\)
      \pvsid{limit}\pvsid{(}\pvsid{series}\pvsid{(}\(\lambda\) \(i\): \pvsid{limit}\pvsid{(}\pvsid{series}\pvsid{(}\(\lambda\) \(j\): \(u\)\pvsid{(}\(i\), \(j\)\pvsid{)}\pvsid{)}\pvsid{)}\pvsid{)}\pvsid{)}\vspace*{\pvsdeclspacing}

  \pvsid{double\_right\_convergence}: \pvskey{LEMMA}
    \pvsid{convergence}\pvsid{(}\pvsid{series}\pvsid{(}\pvsid{single\_index}\pvsid{(}\(u\)\pvsid{)}\pvsid{)}, \(l\)\pvsid{)} \(\equiv\)
     \pvsid{convergence}\pvsid{(}\pvsid{series}\pvsid{(}\pvsid{single\_index}\pvsid{(}\(\lambda\) \(j\), \(i\): \(u\)\pvsid{(}\(i\), \(j\)\pvsid{)}\pvsid{)}\pvsid{)}, \(l\)\pvsid{)}\vspace*{\pvsdeclspacing}

  \pvsid{double\_right\_convergent}: \pvskey{LEMMA}
    \pvsid{convergent}\pvsid{(}\pvsid{series}\pvsid{(}\pvsid{single\_index}\pvsid{(}\(u\)\pvsid{)}\pvsid{)}\pvsid{)} \(\equiv\)
     \pvsid{convergent}\pvsid{(}\pvsid{series}\pvsid{(}\pvsid{single\_index}\pvsid{(}\(\lambda\) \(j\), \(i\): \(u\)\pvsid{(}\(i\), \(j\)\pvsid{)}\pvsid{)}\pvsid{)}\pvsid{)}\vspace*{\pvsdeclspacing}

  \pvsid{double\_right\_limit}: \pvskey{LEMMA}
    \pvsid{convergent}\pvsid{(}\pvsid{series}\pvsid{(}\pvsid{single\_index}\pvsid{(}\(u\)\pvsid{)}\pvsid{)}\pvsid{)} \(\Rightarrow\)
     \pvsid{limit}\pvsid{(}\pvsid{series}\pvsid{(}\pvsid{single\_index}\pvsid{(}\(u\)\pvsid{)}\pvsid{)}\pvsid{)} \(=\)
      \pvsid{limit}\pvsid{(}\pvsid{series}\pvsid{(}\pvsid{single\_index}\pvsid{(}\(\lambda\) \(j\), \(i\): \(u\)\pvsid{(}\(i\), \(j\)\pvsid{)}\pvsid{)}\pvsid{)}\pvsid{)}\vspace*{\pvsdeclspacing}

 \pvskey{END} \pvsid{double\_nn\_sequence}\end{alltt}


\end{document}